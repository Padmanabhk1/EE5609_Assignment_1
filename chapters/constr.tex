\renewcommand{\theequation}{\theenumi}
\begin{enumerate}[label=\thesection.\arabic*.,ref=\thesection.\theenumi]
\numberwithin{equation}{enumi}
%\item The figure for the parallelogram obtained in the question looks like Fig. \ref{fig:parallelogram1}.
%%with angles $\phase{ A},\phase{ C}$ and $\phase{ B}$ and sides $a, b$ and $c$.  The unique feature of this triangle is $\phase{ C}$ which is defined to be $90\degree$.
%with with vectors \textbf{a} and \textbf{b}.


%\renewcommand{\thefigure}{\theenumi.\arabic{figure}}
%\begin{figure}[!ht]
%\centering
%\resizebox{\columnwidth}{!}{\input{./figs/parallelo.tex}}
%\caption{Parallelogram by Latex-Tikz}
%\label{fig:parallelogram1}	
%\end{figure}
%
%
%\renewcommand{\thefigure}{\theenumi}
%
\item List the design parameters for construction
\label{const:table1}
\\
\solution See Table. \ref{table:table1}. 
%
\begin{table}[ht!]
\centering
\begin{tabular}{ |p{3cm}|p{3cm}|  }
%\hline
% \multicolumn{2}{|c|}{Initial Input Values.} \\
\hline
Parameters & Values \\
\hline
%OA (a) & 4\\
$\vec{P} $ & $$\begin{pmatrix}1\\-3\\4\end{pmatrix} $$\\
\hline
$\vec{Q}$ & $$\begin{pmatrix}-4\\1\\2\end{pmatrix}  $$\\
\hline

%$\phase{(ACB)$ & $90^{\circ}$ \\
%\hline
\end{tabular}
%\input{./tables/inp.tex}
\caption{Values of the points}
\label{table:table1}	
\end{table}

\item Generating the points and distance between them using python.\\
\solution The  following Python code generates Fig. \ref{fig:point_distance}
\begin{lstlisting}
codes/point_distance.py
\end{lstlisting}

\begin{figure}[!ht]
\centering
\includegraphics[width=\columnwidth]{./figs/point_distance.png}
\caption{Two points and distance between them.}
\label{fig:point_distance}
\end{figure}


%$\because \vec{P}$ is the midpoint of $CA$,
%\begin{align}
%\vec{P}= \frac{\vec{C}+\vec{A}}{2} = \frac{1}{2}\myvec{9\\-4}
%\label{eq:constr_p}
%\end{align}
%
%Also, $\vec{M}$ is given to be the midpoint of $C$.  Hence, 
%\begin{align}
%\vec{M}&= \frac{\vec{C}+\vec{D}}{2}
%\\
%\implies \vec{D} &= 2 \vec{M} - \vec{C} = \myvec{a\\b}
%\label{eq:constr_d}
%\end{align}
%
%The values are listed in Table. \ref{table:table2} 
%\item List the  derived values.
%\label{const:table2}
%\\
%\solution See  
%Table. \ref{table:table2} 
%\begin{table}[ht!]
%\centering
%\begin{tabular}{ |p{3cm}|p{3cm}|  }
%\hline
% \multicolumn{2}{|c|}{Derived Values.} \\
%\hline
%$\vec{O}$ & $$\begin{pmatrix}0\\0\\0\end{pmatrix} $$\\
%\hline
%$\vec{A}$ & $$\begin{pmatrix}3\\1\\4\end{pmatrix}$$\\						
%\hline
%$\vec{B}$ & $$\begin{pmatrix}1\\-1\\1\end{pmatrix} $$\\
%\hline
%$\vec{C}$ & $$\begin{pmatrix}4\\0\\5\end{pmatrix} $$\\
%\hline
%\end{tabular}
%\caption{To get the vertices of the Parallelogram $OACB$}
%\label{table:table2}
%\end{table}
%
\item Verification of the solution by using python code	
\\
\solution The  following Python code verifies the solution.
%
\begin{lstlisting}
codes/verify_distance.py
\end{lstlisting}

%The  following Python code verifies the cross-product value.
%
%\begin{lstlisting}
%codes/cross_product_check.py
%\end{lstlisting}

%\begin{figure}[!ht]
%\centering
%\includegraphics[width=\columnwidth]{./figs/parallelogram.png}
%\caption{Parallelogram generated using python 3D-plot}
%\label{fig:paral_sss_py}
%\end{figure}

%
%and the equivalent latex-tikz code generating Fig. \ref{fig:parallelogram1} is 
%\begin{lstlisting}
%figs/parallelo.tex
%\end{lstlisting}
%
%The above latex code can be compiled as a standalone document as
%\begin{lstlisting}
%figs/triangle_fig.tex
%\end{lstlisting}

%

%

%
%

\end{enumerate}


