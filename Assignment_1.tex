\documentclass[journal,12pt,twocolumn]{IEEEtran}
%

\usepackage{setspace}
\usepackage{gensymb}
\singlespacing

\usepackage{amsmath}
\usepackage{amsthm}
\usepackage{txfonts}
\usepackage{cite}
\usepackage{enumitem}
\usepackage{mathtools}
\usepackage{listings}
    \usepackage{color}                                            %%
    \usepackage{array}                                            %%
    \usepackage{longtable}                                        %%
    \usepackage{calc}                                             %%
    \usepackage{multirow}                                         %%
    \usepackage{hhline}                                           %%
    \usepackage{ifthen}                                           %%
  %optionally (for landscape tables embedded in another document): %%
    \usepackage{lscape}     
\usepackage{multicol}
\usepackage{chngcntr}
\usepackage{tikz}
\usepackage{pgfplots}
\renewcommand\thesection{\arabic{section}}
\renewcommand\thesubsection{\thesection.\arabic{subsection}}
\renewcommand\thesubsubsection{\thesubsection.\arabic{subsubsection}}

\renewcommand\thesectiondis{\arabic{section}}
\renewcommand\thesubsectiondis{\thesectiondis.\arabic{subsection}}
\renewcommand\thesubsubsectiondis{\thesubsectiondis.\arabic{subsubsection}}

% correct bad hyphenation here
\hyphenation{op-tical net-works semi-conduc-tor}
\def\inputGnumericTable{}                                 %%

\lstset{
%language=C,
frame=single, 
breaklines=true,
columns=fullflexible
}

\begin{document}
%


\newtheorem{theorem}{Theorem}[section]
\newtheorem{problem}{Problem}
\newtheorem{proposition}{Proposition}[section]
\newtheorem{lemma}{Lemma}[section]
\newtheorem{corollary}[theorem]{Corollary}
\newtheorem{example}{Example}[section]
\newtheorem{definition}[problem]{Definition}
\newcommand{\BEQA}{\begin{eqnarray}}
\newcommand{\EEQA}{\end{eqnarray}}
\newcommand{\define}{\stackrel{\triangle}{=}}
\bibliographystyle{IEEEtran}
\providecommand{\mbf}{\mathbf}
\providecommand{\pr}[1]{\ensuremath{\Pr\left(#1\right)}}
\providecommand{\qfunc}[1]{\ensuremath{Q\left(#1\right)}}
\providecommand{\sbrak}[1]{\ensuremath{{}\left[#1\right]}}
\providecommand{\lsbrak}[1]{\ensuremath{{}\left[#1\right.}}
\providecommand{\rsbrak}[1]{\ensuremath{{}\left.#1\right]}}
\providecommand{\brak}[1]{\ensuremath{\left(#1\right)}}
\providecommand{\lbrak}[1]{\ensuremath{\left(#1\right.}}
\providecommand{\rbrak}[1]{\ensuremath{\left.#1\right)}}
\providecommand{\cbrak}[1]{\ensuremath{\left\{#1\right\}}}
\providecommand{\lcbrak}[1]{\ensuremath{\left\{#1\right.}}
\providecommand{\rcbrak}[1]{\ensuremath{\left.#1\right\}}}
\theoremstyle{remark}
\newtheorem{rem}{Remark}
\newcommand{\sgn}{\mathop{\mathrm{sgn}}}
\providecommand{\abs}[1]{\left\vert#1\right\vert}
\providecommand{\res}[1]{\Res\displaylimits_{#1}} 
\providecommand{\norm}[1]{\left\lVert#1\right\rVert}
\providecommand{\mtx}[1]{\mathbf{#1}}
\providecommand{\mean}[1]{E\left[ #1 \right]}
\providecommand{\fourier}{\overset{\mathcal{F}}{ \rightleftharpoons}}
\providecommand{\system}{\overset{\mathcal{H}}{ \longleftrightarrow}}
\newcommand{\solution}{\noindent \textbf{Solution: }}
\newcommand{\cosec}{\,\text{cosec}\,}
\providecommand{\dec}[2]{\ensuremath{\overset{#1}{\underset{#2}{\gtrless}}}}
\newcommand{\myvec}[1]{\ensuremath{\begin{pmatrix}#1\end{pmatrix}}}
\newcommand{\cmyvec}[1]{\ensuremath{\begin{pmatrix*}[c]#1\end{pmatrix*}}}
\newcommand{\mydet}[1]{\ensuremath{\begin{vmatrix}#1\end{vmatrix}}}
\newcommand{\proj}[2]{\textbf{proj}_{\vec{#1}}\vec{#2}}
\let\StandardTheFigure\thefigure
\let\vec\mathbf
\title{Assignment - 1}
\author{Padmanabh Khoptikar
\\ MD/2020/708}
% make the title area
\maketitle
\newpage
%\tableofcontents
\bigskip
\renewcommand{\thefigure}{\theenumi}
\renewcommand{\thetable}{\theenumi}
%\renewcommand{\theequation}{\theenumi}
\begin{abstract}
This is a simple document to learn about writing vectors and matrices using latex, draw figures using Python, Latex.
\end{abstract}
%Download all python codes 
%
%\begin{lstlisting}
%svn co https://github.com/JayatiD93/trunk/My_solution_design/codes
%\end{lstlisting}
Download all and latex-tikz codes from 
%
\begin{lstlisting}
svn co https://github.com/arjunjc93/Assignment-1_new.git
\end{lstlisting}
%
\section{Vectors\\(CBSE-Math-X-2006-Set 1-Q.1)}
\renewcommand{\theequation}{\theenumi}
\begin{enumerate}[label=\thesection.\arabic*.,ref=\thesection.\theenumi]
\numberwithin{equation}{enumi}
\item Solve for x and y: 
\begin{align}
    47x+31y=63 \\or \myvec{47&31}\vec{x}=63\\
31x+47y=15 \\or \myvec{31&47}\vec{x}=15 
\end{align} 
\\
\solution\begin{enumerate}
    \item We have two equations:
    Which is written in vector form:
    \begin{align}\label{eq:1}
        \myvec{47&31}\vec{Z}=63
    \end{align}\\
    and 
    \begin{align}\label{eq:2}
        \myvec{31&47}\vec{Z}=15
    \end{align}
\\where \begin{align}
        \vec{Z}=\myvec{x\\y}
    \end{align}

Both equations are written together in matrix form as:
\begin{align}
    \myvec{47 & 31 \\ 31 & 47}\vec{Z}=
\myvec{63 \\ 15}
\end{align}
Augmented matrix for above is:
\begin{align}
    \myvec{47 & 31 & 63\\
           31 & 47 & 15}
\end{align}
This can be reduced as follows:
 \begin{align}
     \myvec{47 & 31 & 63\\
           31 & 47 & 15}
          \xleftrightarrow{R_1 \leftarrow \frac{R_1}{47}}
    \myvec{1 &\frac{31}{47}&\frac{63}{47}\\
        31&47&15}\\
        \xleftrightarrow{R_2\leftarrow R_2-31R_1}
    \myvec{1&\frac{31}{47}&\frac{63}{47}\\0&\frac{1248}{47}&\frac{-1248}{47}}\\
    \xleftrightarrow{R_2\leftarrow \frac{47}{1248}R_2}
    \myvec{1&\frac{31}{47}&\frac{63}{47}\\0&1&-1}\\
    \xleftrightarrow{R_1\leftarrow R_1-\frac{31}{47}R_2}
    \myvec{1&0&2\\0&1&-1}
 \end{align}
\begin{align}
\therefore \vec{P}=\myvec{2\\-1}
\end{align}
is the solution for x and y for the two equations.
\end{enumerate}
\end{enumerate}
\end{document}